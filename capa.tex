%================================================================================================
%================================= PRIMEIRA FOLHA INTERNA  ======================================
%================================================================================================
\vspace*{2.0cm}
\begin{center}
\large Roger Leonardo Garay Avenda\~no
\end{center}


\vspace*{6.8cm}

\begin{center}
{\sc \Large   Feixes Escalares e Eletromagnéticos não Paraxiais: \\ 
 }{\sc \Large   Uma formulação analítica exata. \\ 
 }
\end{center}

\vspace*{3.25cm}


\null \vfill

\begin{center}
Campinas\\2012
\end{center}
\newpage

%================================================================================================
%====================================== FOLHA DE ROSTO ==========================================
%================================================================================================

\begin{center}
\large Universidade Estadual de Campinas\\
Faculdade de Engenharia Elétrica e de Computação
\end{center}

\vspace*{1.5cm}
\begin{center}
\large Roger Leonardo Garay Avenda\~no
\end{center}


\vspace*{2.3cm}

\begin{center}
{\sc \Large   Feixes Escalares e Eletromagnéticos não Paraxiais: \\ 
 }{\sc \Large   Uma formulação analítica exata. \\ 
 }
\end{center}

\vspace*{3.0cm}

\begin{flushright}
\begin{minipage}{9.0cm}
Texto apresentado \`a Faculdade de Engenharia El\'etrica e de Computa\c{c}\~ao  da Universidade Estadual de Campinas (UNICAMP) para o Exame de Qualifica\c{c}\~ao como parte dos requisitos exigidos para a obten\c{c}\~ao do T\'itulo de Mestre em Engenharia El\'etrica.\\ \'Area de concentra\c{c}\~ao: Telecomunica\c{c}\~oes. 

\vspace*{0.5cm}
Orientador: Michel Zamboni Rached

\end{minipage}
\end{flushright}

\null \vfill
%\begin{minipage}{7cm}
%\small
%Este exemplar corresponde � vers�o final da tese defendida pelo
%aluno, e orientada pelo Prof. Dr. Pedro Luis Dias Peres\\[4mm]
%\rule{7.0cm}{0.2mm} \hfill 
%\end{minipage}

\vspace*{0.5cm}

\begin{center}
Campinas\\2012
\end{center}

%================================================================================================
%============================== Ficha (Somente na vers�o final) =================================
%================================================================================================
%\newpage

%\begin{center}
%\vspace*{10cm}
%Insira nesta p�gina a sua ficha catalogr�fica (somente vers�o final). Obs. � conveniente
%converter o documento fornecido pela BAE (normalmente .doc) em um arquivo .ps. Para a vers�o
%preliminar da tese (antes da defesa), simplesmente remova essa p�gina.
%\end{center}
% Observa��o: a ficha fornecida pela BAE normalmente � fornecida em formato .doc. Existem diversas
% maneiras de converter .doc em .ps. 

% Descomente as duas pr�ximas linhas (e comente acima desde o begin{center} at� o end{center}) para inserir a ficha catalogr�fica caso a mesma j�  tenha sido convertida para .ps (no caso ficha.ps)

%\epsfxsize=0.925\columnwidth
%\epsffile{ficha.ps}

%\null \vfill
%\newpage

%================================================================================================
%============================== Folha de aprova��o (Somente na vers�o final) ====================
%================================================================================================


%\begin{center}
%\vspace*{10cm}
%Insira nesta p�gina a folha de aprova��o fornecida pelo seu programa de p�s-gradua��o (somente vers�o final). 
%Obs. � conveniente scanear o documento e convert�-lo para o formato .ps. Para a vers�o
%preliminar da tese (antes da defesa), simplesmente remova essa p�gina.
%\end{center}

% Escanear a folha de aprova��o fornecida pela CPG e converter para .ps ou .eps. A id�ia
% � inserir como uma figura. � muito prov�vel que ser� necess�rio fazer ajustes no
% tamanho, mexendo no comando \epsfxsize

% Descomente as duas pr�ximas linhas (e comente acima desde o \begin{center} at� o \end{center})
%\epsfxsize=0.985\columnwidth
%\hspace*{-1.25cm}\epsffile{aprov.eps}

%\null \vfill
%\newpage
