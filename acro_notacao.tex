%============================Acrônimos e Notação =================================================

\chapter*{Lista de Acrônimos e Notação}

\begin{tabular}{ll}
TE  & Transversal Elétrico\\
TM & Transversal Magnético\\
%LMI  & Linear Matrix Inequality (desigualdade matricial linear)\\
%LFT  & Linear Fractional Transformation (transformação linear fracionária)\\
%LPV  & Linear Parameter-Varying (linear com parâmetros variantes)\\
%IQC  & Integral Quadratic Constraint (restrição de integral quadrática)\\
\end{tabular}

\vspace*{1cm}

\begin{tabular}{ll}
$1D, 2D, 3D$ & indica dimensão espacial\\
$ x, y, z$ & variáveis espaciais cartesianas\\
$ \rho, \phi, z$ & variáveis espaciais cilíndricas\\
$\lambda$ & comprimento de onda no vácuo\\
$\omega$ & frequência angular de onda no vácuo\\
$c$ & velocidade da luz no vácuo\\
$J_0$ & função de Bessel de ordem zero\\
$J_{\nu}$ & função de Bessel de ordem$-\nu$\\
%$\star$ & indica bloco simétrico nas LMIs\\
%$L > 0$ & indica que a matriz $L$ é simétrica definida positiva\\
%$L \geq 0$ & indica que a matriz $L$ é simétrica semi-definida positiva\\
%$A$ & notação para matrizes (letras maiúsculas do alfabeto latino)\\
%$A'$ & ($'$), pós-posto a um vetor ou matriz, indica a operação de transposição\\
%$\reais$ & conjunto dos números reais\\
%$\mathbb{Z}$ & conjunto dos números inteiros\\
%$\mathbb{Z}_+$ & conjunto dos números inteiros não negativos\\
%$\mathbb{N}$ & conjunto dos números naturais (incluindo o zero)\\
%$\I$ & matriz identidade de dimensão apropriada\\
%$\Z$ & matriz de zeros de dimensão apropriada\\
%$g!$ & símbolo (!), denota fatorial, isto é, $g!=g (g-1) \cdots (2) (1)$ para $g \in \mathbb{N}$\\
$R_n$ & n-ésimo coeficientes de Fourier\\
$2N+1$ & número de coeficientes de Fourier\\
$k_{\rho}$ & componente transversal do vetor de onda\\
$k_{z}$ & componente longitudinal do vetor de onda\\
$S(k_z)$ & espectro em $k_z$\\
$\Delta k_z$ & largura de banda do espectro $S(k_z)$\\
$\bar{k}_{z}$ & posição do centro do espectro $S(k_z)$\\
$\Re e$ & parte real\\
$E_m$ & componentes do campo elétrico do modo $TM$\\
$B_m$ & componentes do campo magnético do modo $TM$\\
$E_e$ & componentes do campo elétrico do modo $TE$\\
$B_e$ & componentes do campo magnético do modo $TE$\\

%$n$ & especialmente utilizada para representar a ordem uma matriz quadrada\\
%$\simplex$ & simplex unitário de $N$ variáveis\\
%$\alpha$ & especialmente utilizada para representar as incertezas de um sistema
\end{tabular}

%==============================================================================================
