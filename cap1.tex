\chapter{Introdução Geral}
\pagenumbering{arabic}

Nesta seção apresentamos  uma revisão bibliográfica resumida, os objetivos e as justificativas  de nosso projeto.
\section{Revisão bibliográfica resumida}
Um feixe é uma solução monocromática da equação de onda que possui concentração transversal de campo. No caso escalar essas soluções são soluções da equação de onda e no caso vetorial essas soluções são soluções das equações de Maxwell.\\
Um feixe óptico pode ser construído a partir de superposições de ondas planas que se propagam em diferentes direções, sendo que o feixe óptico mais comum é o gaussiano.\\
Na atualidade o estudo de feixes ópticos vem apresentando uma grande demanda (teórica e prática), atingindo as mais diversas áreas, como em pinças ópticas 
%****************************************************************************************************************
%****************************************************************************************************************
\section{Objetivos e justificativas}
Este trabalho tem por objetivo geral desenvolver um método teórico capaz de fornecer feixes escalares e vetoriais propagantes não paraxiais com simetria azimutal como soluções analíticas exatas da equação de onda e das equações de Maxwell respectivamente. Feixes escalares propagantes não paraxiais sem simetria azimutal também são estudados como soluções analíticas exatas da equação de onda.\\
Este trabalho tem como objetivos:
\begin{itemize}
\item Adquirir conhecimento dos feixes escalares propagantes não paraxiais com simetria azimutal construídos como superposições de feixes de Bessel de ordem zero através de uma integração na componente longitudinal do vetor de onda $k_z$ com uma dada função espectral $S(k_z)$.
\item Usar o método matemático proposto em  para obter feixes escalares propagantes com simetria azimutal paraxiais e não paraxiais a partir de funções de espectro paraxial e não paraxial. Analisar espectros de tipo exponencial, quadrado e gaussiano. Aplicar o método matemático proposto em  para construir feixes escalares não propagantes sim simetria azimutal. 
\item Propor métodos matemáticos para a construção de feixes vetoriais propagantes não paraxiais simétricos como soluções analíticas exatas das equações de Maxwell. Estudo das polarizações dos feixes vetoriais em coordenadas cartesianas e cilíndricas.
\end{itemize}  
Dentro das justificativas podemos destacar que o estudo de feixes ópticos nos últimos anos vem apresentando uma grande demanda (teórica e prática) principalmente na construção de novos tipos de feixes, os quais possuem propriedades muito mais interessantes que os já conhecidos feixes gaussianos, como as aplicações em fibra óptica, guiamento óptico de átomos, alinhamento óptico, aplicações médicas, comunicações ópticas no espaço livre, aplicações em óptica não linear, etc. O estudo da propagação de feixes é realizado, em geral, a partir das soluções da equação de onda. Na atualidade soluções analíticas exatas da equação de onda descrevendo feixes são muito raras, por isso muita atenção é dada ao regime de propagação paraxial na obtenção de soluções analíticas e também numéricas.\\
Porém obter um método que forneça soluções analíticas exatas da equação de onda no regime não paraxial é de grande importância, pois poderemos tratar situações que não podem ser descritas por aproximações usuais.\\
Dentro das justificativas podemos destacar também que nosso método \cite{Stratton:02} distingue-se dos outros métodos em dois principais características: A primeira é que nosso método pode ser aplicado para qualquer tipo de feixe e não s\'o para um determinado feixe, como em \cite{Stratton:02, Jackson:03,Figueroa:05,Sochacki:1}.
  
***********************************************************************************************************  


Nesta seção apresentamos  uma revisão bibliográfica resumida, os objetivos e as justificativas  de nosso projeto.
\section{Revisão bibliográfica resumida}
Um feixe é uma solução monocromática da equação de onda que possui concentração transversal de campo. No caso escalar essas soluções são soluções da equação de onda e no caso vetorial essas soluções são soluções das equações de Maxwell.\\
Um feixe óptico pode ser construído a partir de superposições de ondas planas que se propagam em diferentes direções, sendo que o feixe óptico mais comum é o gaussiano.\\
Na atualidade o estudo de feixes ópticos vem apresentando uma grande demanda 
