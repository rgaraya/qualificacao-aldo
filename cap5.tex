\chapter{Plano de trabalho e cronograma}
O propósito deste capítulo é apresentar as etapas necessárias para a realização desde trabalho.\\
O cronograma de atividades com resolução bimestral é mostrado na Tabela (\ref{tab1}), o qual descreve a programação das atividades já realizadas e as previstas até o término da dissertação. O número $1$ marca a data da atividade e o número $0$ indica que não há atividade. O número 1 riscado indica que a atividade j foi realizada.\\

\begin{table}[h]
\caption{Cronograma de actividades} %title of the table
\centering
% centering table
\begin{tabular}{c rrrrrr}
% creating eight columns
\hline\hline
%inserting double-line
 &\multicolumn{6}{c}{2012} \\ [0.5ex]
\hline
% inserts single-line
 Temas / Tempo & Jan/Fev & Mar/Abr & Mai/Jun & Jul/Ago & Set/Out & Nov/Dez\\
\hline
\centering
% Entering row contents
Pesquisa bibliogr\'afica & \cancel{1} & 0 & 0 & 0 & 0 & 0\\
Feixes escalares  & \cancel{1} & \cancel{1} & 0 & 0 & 0 & 0\\
Feixes sim simetria & 0 & \cancel{1} & \cancel{1} & 0 & 0 & 0\\
Analice vetorial Met 1&  0 & 0 & \cancel{1} & 1 & 0 & 0\\
Analice vetorial Met 2&  0 & 0 & 0 & 0 & 1 & 0\\
Prepara\c{c}\~ao da diserta\c{c}\~ao &  0 & 0 & 0 & 0 & 1 & 1\\[1ex] % [1ex] adds vertical space
\hline

% inserts single-line
\end{tabular}
\label{tab1}
\end{table}
Iniciamos o plano de trabalho com a pesquisa bibliográfica, a qual também inclui o estudo de feixes escalares e vetoriais propagantes paraxiais e não paraxiais com simetria azimutal os quais foram mostrados no capítulo 2. Neste período estudou-se a metodologia matemática proposta em \cite{Lya:2}, desenvolvida para fornecer soluções analíticas exatas da equação de onda. O método matemático desenvolvido em (\cite{Lya:2}) para a obtenção de feixes escalares não paraxiais sem simetria azimutal é também abordado neste período.\\
Na Tabela (\ref{tab1}) mostramos a continuação do plano de trabalho. A metodologia matemática proposta para a obtenção de feixes escalares e vetoriais foi exposta no capítulo $3$ e $4$, assim como os resultados já obtidos para os feixes escalares com e sem simetria azimutal. No capítulo $4$ também mostramos os resultados obtidos para o caso vetorial mediante o método das derivadas parciais. Nos próximos meses continuaremos com o analise vetorial dos feixes eletromagnéticos com a obtenção de novas e interessantes polarizações.\\
Nosso cronograma ser\'a finalizado com a preparação da dissertação de acordo com as diretrizes do programa de pós-graduação da FEEC-UNICAMP.
\subsection*{Submissões}
At\'e o presente momento, foi realizada a submissão de um artigo para o congresso, o qual foi aceito e estamos finalizando um artigo para ser submetido a um periódico internacional. O artigo submetido e aceito no congresso de MOMAG foi:

\begin{itemize}
\item R. L. Garay-Avendaño and M. Zamboni-Rached, "Feixes Não Paraxiais: Uma Formulação Analítica Exata. I. A Teoria Escalar", aceito no $15^{o}$ Simpósio Brasileiro de Microondas e Optoeletrônica e $10^o$ Congresso Brasileiro de Eletromagnetismo, João Pessoa, PA, Brasil (2012).
\end{itemize}  
